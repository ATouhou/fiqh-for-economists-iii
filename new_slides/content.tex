\section{\textit{Hibah}, \textit{Hadiyah}, and \textit{Ṣadaqah}}\label{hibah-hadiyah-and-ux1e63adaqah}

\begin{frame}{Hibah \hfill \arabicsans{هبة}}

\begin{block}{Definition}
A contract by which a person transfers his property with immediate effect without any consideration to another person.
\end{block}
It is different from \textit{Iarah}.

Can be given through:
\begin{itemize}
\item {\arabicsans{تمليك}} (Transfer of Ownership). Based on the narration {\arabicfont{تَهَادُوا تَحَابُّوا}} (Exchange gifts so that you will love each other), reported by Imam al-Bukhārī in \textit{Adab al-Mufrad}.
\end{itemize}
\end{frame}

\begin{frame}{Hibah \hfill \arabicsans{هبة}}
\begin{itemize}
  \item {\arabicsans{إسقاط}} (Cancellation of a Debt). Based on the \textit{āyah},\\{\arabicsans{...فَإن طِبْنَ لَكُمْ عَنْ شَيْءٍ مِّنْهُ نَفْساً فَكُلُوهُ هَنِيئاً مَّرِيئاً}}\\
  “If they, of their own good pleasure, remit any part of it to you, take it and enjoy it with right good cheer.”\\\hfill(\textit{Sūrah al-Nisā}, 4)
\end{itemize}
\end{frame}

\begin{frame}{Pillars of \textit{Hibah} \hfill \arabicsans{أركان الهبة}}
\textbf{Ḥanafīs} Offer and Acceptance only.

\textbf{Majority} There are four pillars.

\begin{enumerate}
  \item {\arabicsans{واهب}} (Donor)
  \item (Donee) {\arabicsans{موهوب له}}
  \item {\arabicsans{موهوب}} (Gift). Complete ownership, existence (unless if it is a debt expected in the future), capability of delivering it, specific, possession: “Gift is not valid except with possession”; permission to possess (except Mālikīs), directly or through representatives.
  \item {\arabicsans{صيغة}} (Expression). Explicitly, implicitly, or metaphorically.
\end{enumerate}
\end{frame}

\begin{frame}{Types of \textit{Hibah} \hfill {\arabicsans{أقسام الهبة}}}
\begin{enumerate}
\item {\arabicsans{أأأ}} A gift with consideration.
\begin{itemize}
\item Equivalent to sale, therefore, options of inspection and option of defect can be practiced.  
\item Once gifts are exchanged between the two parties the contract is valid and cannot be revoked.
\end{itemize}
\item {\arabicsans{مشاء}} “Joint share in undivided property subject to the right of more than one individual” (ex: horse, car, machineries..)
\begin{itemize}
\item \textbf{Ḥanafīs} If the property is divisible the gift is void or irregular ({\arabicsans{فاسد}}) until separated.
\item \textbf{Majority} It is valid.
\end{itemize}
\end{enumerate}
\end{frame}

\begin{frame}
\begin{enumerate}
  \setcounter{enumi}{2}
\item Gift of joint shares in favor of heirs.
\begin{itemize}
\item \textbf{Ḥanafīs} Gift of a property--–whether divisible or not---to an heir is valid but voidable to a stranger if the property is divisible.
\item \textbf{Majority} Valid in all cases.
\end{itemize}
\end{enumerate}
\end{frame}

\begin{frame}[standout]
A gift by a bankrupt is \alert{INVALID} if intended to defraud the creditors.
\end{frame}

\begin{frame}{Gift to Heirs}
Parents should observe equality in giving gifts to their children, whether male of female.
\begin{block}{Ḥadīth \hfill Al-Bukhārī and Muslim}
Nuʿmān b. Bashīr reported: My father donated to me some of his property. My mother Amra bint Rawaha said, “I shall not be pleased (with this act) until you make Allah's Messenger \pbuh a witness to it.” My father went to Allah's Apostle \pbuh in order to make him the witness of the donation given to me. Allah's Messenger \pbuh said to him, “Have you done the same with every son of yours?” He said: “No.” Thereupon he \pbuh said: “Fear Allah, and observe equity in case of your children.” My father returned and got back the gift.
\end{block}
\end{frame}

\begin{frame}
Is equality obligatory or recommendable?

\textbf{Ḥanafīs}, \textbf{Shāfiʿīs}, and some \textbf{Mālikīs} \alert{RECOMMENDABLE}
\begin{itemize}
\item A Muslim---in his lifetime and health---has the right to give as a gift his entire property to whomsoever he likes, be it an heir or a stranger.
\item While it is legally permissible, it is not virtuous to do so, for it might be a sin in many cases.
\item However, preferential gifts to the elder siblings is permissible.
\end{itemize}
\end{frame}

\begin{frame}
\textbf{Imam Aḥmad}, \textbf{al-Thawrī}, and \textbf{some others} \alert{OBLIGATORY}
\begin{itemize}
\item If equality is not observed, the gift is \textbf{VOID} based on the ḥadīth.
\item However, Imam Aḥmad allowed a preferential gift in \textbf{some} cases like being \alert{crippled}, \alert{indebted}, or \alert{unable to earn} like other siblings.
\item Similar Fatwā was given by al-Azhar's Department of Fatwa. [citation needed]

\end{itemize}
\end{frame}

\begin{frame}{Revocation of Gifts}
It is extremely disliked in Islam to revoke the gift after possession by the donee.

\begin{block}{Ḥadīth \hfill Al-Bukhārī and Muslim}
One who gets back his gift is like a dog which vomits and then swallows that vomit.
\end{block}

\textbf{Majority} The donor has right to revoke his gift at any time \textbf{before} the donee possesses it, however, it is \alert{unlawful} to revoke the gift \textbf{after} its possession.

If either donor or the donee dies \textbf{before} the possession the gift will be included in the estate of the \alert{donor}.
\end{frame}

\subsection{Ṣadaqah}

\begin{frame}{Charity \hfill \arabicsans{صدقة}}
Similar to gifts, possession is a necessity in charity.

Allāh ({\arabicsans{سُبْحَانَهُ وَتَعَالَىٰ}}) says,\\[10pt]

\begin{Arabic}
الَّذِينَ يُنفِقُونَ أَمْوَالَهُمْ فِي سَبِيلِ اللّهِ ثُمَّ لاَ يُتْبِعُونَ مَا أَنفَقُواُ مَنّاً وَلاَ أَذًى لَّهُمْ أَجْرُهُمْ عِندَ رَبِّهِمْ وَلاَ خَوْفٌ عَلَيْهِمْ وَلاَ هُمْ يَحْزَنُونَ
\end{Arabic}

“Those who spend their substance in the cause of Allah, and follow not up their gifts with reminders of their generosity or with injury,-for them their reward is with their Lord: on them shall be no fear, nor shall they grieve.” (\textit{Sūrah al-Baqarah}, 262)
\end{frame}

\begin{frame}
\begin{block}{Ḥadīth \hfill Al-Bukhārī and Muslim}
ʿUmar bin al-Khaṭṭāb said:
“I gave a horse in Allāh's Cause. The person to whom it was given, did not look after it. I intended to buy it from him, thinking that he would sell it cheap. When I asked the Prophet \pbuh, he said, ‘Don't buy it, even if he gives it to you for one Dirham, as the person who takes back what he has given in charity, is like a dog that swallows back its vomit.’”
\end{block}
\end{frame}

\begin{frame}
\begin{itemize}
\item Charity in contrast to gift can not be revoked.
\item The purpose of charity is to seek Allāh’s pleasure while gift and present might be given for the sake of respect, love or affection, or even without any motive.
\item Charity is given to the needy while gift can be given to the needy and others.
\end{itemize}
\end{frame}

\section{\textit{Waqf} \hfill \arabicsans{وقف}}

\begin{frame}{\textit{Waqf} \hfill \arabicsans{وقف}}
\begin{block}{Definition}
A permanent dedication of the corpus of the valuable property by a Muslim to the ownership of Allah ({\arabicsans{سُبْحَانَهُ وَتَعَالَىٰ}}) for religious, charitable and pious purposes.
\end{block}
\end{frame}

\begin{frame}
\begin{itemize}
\item  \textit{Waqf} property can't be sold, mortgaged, donated or alienated even by inheritance.
\item In \textit{waqf} unlike the \textit{ṣadaqah} the corpus of the property can not be consumed, only income of the endowed property can be consumed.
\item \textit{Waqf} can be created for the support of the founder’s own immediate descendants for specific time or in perpetuity. 
\item In case of difficulty to realize the very purpose of the \textit{waqf} the property will not revert to the founder’s descendants. The principle of perpetuity continues.
\end{itemize}
\end{frame}

\begin{frame}{\textit{Waqf} Legality}
\begin{block}{Ḥadīth \hfill Al-Tirmidhī}
The Prophet \pbuh said to ʿUmar:
“Tie up the property and devote the usufruct to human beings, and it is not to sold or made the subject of gift or inheritance, devote its produce to your children, and the poor in the way of Allāh.”
\end{block}
\begin{block}{Ḥadīth \hfill Muslim}
When a son of Ādam dies, his good deeds come to an end except for three: continuous charity, a knowledge from which some benefit may be obtained, and a pious child who prays for him.”
\end{block}
\end{frame}

\begin{frame}{Pillars of \textit{Waqf} \hfill \arabicsans{أركان الوقف}}
\textbf{Ḥanafīs} Offer only.

\textbf{Majority} There are four pillars.

\begin{enumerate}
\item {\arabicsans{واقف}} (The Founder)
\item {\arabicsans{موقوف}} (\textit{Waqf} property)
\item {\arabicsans{موقوف عليه}} (Beneficiaries)
\item {\arabicsans{صيغة}} (Expression). Includes offer and acceptance.
\end{enumerate}
\end{frame}

\begin{frame}{\textit{Waqf} Conditions}
Founder {\arabicsans{واقف}}
\begin{enumerate}
\item Must be Muslim.
\item Have ownership at the time of endowment.
\item Must have legal capacity\\\textit{Waqf} by insane, minor, bankrupt, prodigal (\arabicsans{سفي}) or with intention to deprive the creditor or his own heirs from the estate is \alert{INVALID}.
\end{enumerate}
\end{frame}

\begin{frame}{\textit{Waqf} Conditions}
\arabicsans{موقوف} \textit{Waqf} Property:
\begin{enumerate}
  \item Must be valuable in the eyes of the Shairah and must be ‘ayn (goods, property, assets, etc.).\textit{Waqf} of usufruct is not allowed (disputed issue).
  \item Must exist and be specific.
  \item Money can also be endowed (practiced during the Ottoman era)
  \item Appointment of administrator (\arabicsans{متولي}).
\end{enumerate}
\end{frame}

\begin{frame}{\textit{Waqf} Conditions}
\arabicsans{موقوف عليه} Beneficiaries:
\begin{itemize}
\item Can be specific or undetermined, one or more than one, but for one the acceptance is required according to the majority of schools.
\item It is permissible to dedicate it for relatives or even to dzimmies.
\item It is not valid to endow in favor of synagogues, churches or animals. 
\end{itemize}
\end{frame}

\begin{frame}{\textit{Waqf} Conditions}
\arabicsans{صيغة} Expression:
\begin{itemize}
\item No specific expression or words but the meaning.
\item Perpetual and not for certain period. (Disputed issue).
\item No revocation after declaration is made unless if it was stated in his will and he revoked it before his death.
\item \textit{Waqf} should immediately take effect (no dependence on certain occasion) unless if it was dedicated by will then it will function only upon his death.
\item No void conditions such as: I may cancel the \textit{waqf}, or administrator or his children should not be removed... In this case the conditions are void and  \textit{waqf} is valid.
\end{itemize}
\end{frame}

\begin{frame}{Types of \textit{Waqf}}
\begin{enumerate}
\item Public \textit{waqf}. Made for the public benefit (bridges, mosques, cemeteries, etc.)
\item Quasi-Public \textit{waqf}. Made for the benefit of a particular group or individuals (scholars, students of religious knowledge, etc.)\\
The above two types of \textit{waqf} are called {\arabicsans{خيري}}.
\item \textit{Waqf dhurrī} or \textit{ahlī}. Made for someone’s children, grandchildren and so on. However, it must be done for religious, pious or charitable purpose. And if the shares are not expressly specified then every male and female of the family will get an equal share.
\end{enumerate}
\end{frame}

\begin{frame}{Power and Duties of a \textit{Mutawallī}}
\begin{itemize}
\item \textit{Mutawallī} is just a mere administrator who manages the \textit{waqf} affairs but does not own it. However, he bears a serious responsibility before Allah ({\arabicsans{سُبْحَانَهُ وَتَعَالَىٰ}}), the founder and the beneficiaries if something goes wrong because of his carelessness or irresponsibility.
\item He may be appointed either by the founder during his life time, through his \textit{wasiyyah} or by a \textit{Qāḍī} (Judge).
\item He may specify (a) person(s) who shall be entitled to be appointed as \textit{Mutawallī} after him.
\item He can not give up without permission of \textit{Qāḍī}.
\end{itemize}
\end{frame}

\begin{frame}{Limitation of a \textit{Mutawallī's} Powers}
A \textit{mutawallī} is not allowed to do the following without permission of the court.
\begin{enumerate}
\item To mortgage or change the ownership of \textit{waqf} property.
\item To transfer his duties, functions and powers to another person and make him the trustee.
\item To lease the \textit{waqf} property for more than one year in case of non-agricultural land, and for more than three years in case of agricultural land.
\item To borrow money for spending on beneficiaries.
\end{enumerate}

The court may remove the \textit{mutawallī} if found negligent, careless or irresponsible. 
\end{frame}

\begin{frame}{Maintenance of \textit{Waqf} Property}
\begin{itemize}
\item Debt for renovation of \textit{waqf} property = only with permission of the court and should be returned from the net income of \textit{waqf}.
\item He may suggest a better business plan such as proposing to build houses on the land which initially was meant for agriculture.
\item Neither he nor his relatives are allowed to rent it unless approved by the court.
\item No interference of beneficiaries in \textit{waqf} business is tolerated. It is solely done by the \textit{mutawallī} (rent collection...)
\item If \textit{waqf's} purpose is fulfilled then the extra money can be used to buy other properties, and the latter may be sold in the future if needed.
\end{itemize}
\end{frame}

\begin{frame}{Significance of \textit{Waqf}}
\begin{itemize}
\item Social security and welfare
\item Self-sufficiency and financial independence for Muslim educational institutions (eg: University of Al-Azhar).
\item Protection of Muslims' wealth by prohibiting the sale of \textit{waqf} property.
\item Any activity to expand, enlarge, rebuild or improvise existing \textit{waqf} property.
\end{itemize}
\end{frame}

\begin{frame}{Development of \textit{Waqf}}
\textbf{\textit{Mursad} loan} for dilapidated \textit{waqf} property.
\begin{itemize}
\item Reconstructed \textit{Waqf} property is rented out to the lender until the loan is fully paid plus some profit.
\end{itemize}
\end{frame}

\begin{frame}{Development of \textit{Waqf}}
\textbf{\textit{Hukr}} – Monopoly or Exclusivity\\
\begin{itemize}
  \item It is a lump sum rental paid in advance which is used for \textit{Waqf} purpose (renovation of other \textit{Waqf} properties).
  \item It is usually rented for a long period of time.
  \item \textit{Mursad} and \textit{Hukr} both are meant to protect and maintain the \textit{waqf} property rather than to increase its income.
  \item In some countries the loans were abolished for they were used to convert the \textit{waqf} property into private property.
\end{itemize}
\end{frame}

\begin{frame}{Development of \textit{Waqf}}
\textbf{\textit{Istisnā}}
\begin{enumerate}
\item $Bank + \textit{Waqf}\  institution$\\
\alert{\textit{Mushārakah Mutanāqisah}} Both partners will receive the agreed upon share of the profit. Once the project is completed the equity shares of the bank will be gradually reduced until the \textit{waqf} institution becomes the full owner of the project.
\item $Bank + Construction \ Company$
\end{enumerate}
\end{frame}

\section{\textit{Wasiyyah} (Wills)}

\begin{frame}{\textit{Wasiyyah}}
\begin{block}{Definition}
A transaction that comes into operation after the testator's death.
\end{block}
\end{frame}

\begin{frame}{Legality of \textit{Wasiyyah}}
\begin{enumerate}
\item The Quran states,\\
<Insert Ayah here>\\
<Insert Translation here>
\end{enumerate}
\end{frame}

\begin{frame}{Legality of \textit{Wasiyyah}}
\begin{enumerate}
\setcounter{enumi}{1}
\item ʿAlī (May Allāh be pleased with him) narrated,\\
<Insert Arabic Text>\\
<Insert Translation Text>\\
\alert{NOTE} The will should not contain more than \sfrac{1}{3} of the estate bsaed on what the Prophet \pbuh said to Saʿd ibn Abī Waqqās.
\end{enumerate}

\end{frame}

\begin{frame}{Legality of \textit{Wasiyyah}}
\begin{block}{Ḥadīth \hfill Al-Bukhārī}
“I was stricken by an ailment that led me to the verge of death. The Prophet \pbuh came to pay me a visit. I said, ‘O Allah's Apostle! I have much property and no heir except my single daughter. Shall I give two-thirds of my property in charity?’ He said, ‘No.’ I said, ‘Half of it?’ He said, ‘No.’ I said, ‘One-third of it?’ He said, ‘You may do so, although one-third is also much, for it is better for you to leave your off-spring wealthy than to leave them poor, asking others for help…’” \hfill --- Saʿd ibn Abī Waqqās
\end{block}
\end{frame}

\begin{frame}{Importance of \textit{Wasiyyah}}
It is preferable that the will is made in favor of relatives who are poor and may not inherit.\\[16pt]

\textit{Wasiyyah} \textbf{helps}:
\begin{itemize}
\item people who are not entitled to inherit such as an orphaned grandchild, and a Christian or Jewish widow.
\item to settle debts.
\item appoint a guardian for one's children.
\item settle disputes if intestate succession law is against Islamic law.
\end{itemize}

It is \alert{reprehensible} for a person to make a will for others while \textbf{his heirs} are poor.
\end{frame}

\begin{frame}{Pillars of \textit{Wasiyyah}}
\textbf{Ḥanafīs} Offer only\\
\textbf{Majority}
\begin{enumerate}
\item ({\arabicsans{صيغة}}) Expression
\begin{itemize}
\item Words
\item Writing
\item Gestures (if unable to speak)
\item Two witnesses (unless if handwriting is known)
\end{itemize}
\item ({\arabicsans{موصي}}) Testate
\begin{itemize}
\item Legal capacity
\item Voluntarily
\item For disbelievers and vice versa (if property is not \textit{ḥarām})
\end{itemize}
\end{enumerate}
\end{frame}

\begin{frame}{Pillars of \textit{Wasiyyah}}
\begin{enumerate}
\setcounter{enumi}{2}
\item ({\arabicsans{موصي له}}) Legatee
\begin{itemize}
\item Specific
\item Individual or group, institution or organization
\item Existence (or fetus if it is known at the time of will)
\item Children born after 6 months after the will are not entitled to the bequest
\item Capable of ownership
\item Should not be a murderer
\item If the legatee \textbf{dies before} the testate, the will becomes \alert{invalid}
\end{itemize}
\item ({\arabicsans{موصي به}}) Legacy
\begin{itemize}
\item Valuable property
\item Not more than \sfrac{1}{3} after debts and fun? expenses.
\item May be usufruct for life or definite period.
\end{itemize}
\end{enumerate}
\end{frame}

\begin{frame}{Types of \textit{Wasiyyah}}
\begin{enumerate}
\item (\arabicsans{وصية مطلقة}) Absolute Will
\item (\arabicsans{وصية مقيدة}) Restricted Will.\\
Example: “If I do not survive this illness or die in this country the will is valid otherwise it is invalid, etc.”
\item (\arabicsans{وصية إلزامية أو إجبارية})\\
No legal inheritance for grandchildren whose parents died during the lifetime of their parents. However, if grandfather dies without leaving a will or dies with his son in a car accident then either \sfrac{1}{3} of the estate or the share of his deceased son should be given to them based on presumption that son died after his father died.
\end{enumerate}
\end{frame}

\begin{frame}{Difference between \textit{Hadiyyah} and \textit{Wasiyyah}}
\begin{itemize}
\item Gift is an immediate transfer of property while will takes place after the death of the testator.
\item In gift in contrast to will possession is necessary
\item No limit in gift while the maximum will is \sfrac{1}{3}.
\item Gift in contrast to will can be made for heirs.
\item A possessed gift in contrast to will can not be revoked. Therefore, a testator may revoke his will at any time either by words or by action (selling or buying the property) or by tearing the will document.
\end{itemize}
\end{frame}

\begin{frame}{Administration of the estate}
\begin{block}{Definition}
Management of the property of the deceased for a temporary period. (debts, credits, will, estate, mortgaged, property, guardianship...)
\end{block}
\begin{itemize}
\item Guardianship or \textit{al-Walāyah} (appointed by the judge)\\
It can be over property or ward’s private matters such as upbringing, education, health care, marriage etc.
\item Priority in Guardianship\\
\textbf{Ḥanafīs} argue that the father would not appoint him if wasn't a better choice.\\
\textbf{Shāfiʿīs} Grandfather, then the executor appointed by him or a judge.\\
\textbf{Majority} Executor (\textit{wasi}) appointed by the father, then the judge or an executor appointed by a judge.
\end{itemize}
\end{frame}

\begin{frame}{The Executor \hfill \arabicsans{الوصي}}
\begin{itemize}
\item Can be an heir or a non-heir, male or female, relative or a stranger, one or more. A father may appoint him to look after all his property and entire family members or to manage a certain property, or to look after one of his children.
\item No guardian is needed if all heirs are adults and able to manage the property. However, a minor, a lunatic and an idiot are in need of guardian.
\item A minor in contrast to lunatic and idiot will not need a guardian when he attains the age of majority. But if the lunacy or idiocy appeared after the age of majority then he will have a guardian appointed by judge.
\end{itemize}
\end{frame}

\begin{frame}{The Executor \hfill \arabicsans{الوصي}}
\begin{itemize}
\item A prodigal also needs a guardian but only court may decide whether he is a prodigal or not.
\item A bankrupt does not need a guardianship but restriction.
\end{itemize}
\end{frame}

\begin{frame}{Conditions of Guardianship}
\begin{itemize}
\item A complete capacity to perform the duty.
\item Must have the same religion with the ward.
\item Must be just, pious, and eager to safeguard the interests of the ward.
\item Expenditure from the ward’s property must be moderate according to his social status and wealth.
\item He may sell---if necessary---the movable property of the ward but not the immovable except for his essential needs.
\item No charity from the ward’s property.
\item No selling and buying between him and the ward.
\item In case of violation of one of these conditions the judge will replace him with someone else.
\end{itemize}
\end{frame}

\begin{frame}{Distribution of the Estate}
\begin{enumerate}
\item Funeral Expenses
\item Debts
\item Will
\item Shares of the heirs
\end{enumerate}
\end{frame}

\begin{frame}{Distribution of the Estate -- Funeral Expenses}
\textbf{Funeral Expenses}\\
Should be done in a moderate way without extravagance and deficiency.
\end{frame}

\begin{frame}{Distribution of the Estate -- Debts}
\textbf{Debts}\\
\begin{itemize}
\item The Prophet \pbuh said that even martyrdom---repeated three times---would not atone for undischarged debts.
\item All schools agree that funeral expenses take priority over payable debts.
\item However, they disagreed upon debts which are not payable and whether Allah’s debts should be given priority over people’s debts or not.
\end{itemize}
\end{frame}

\begin{frame}{Distribution of the Estate -- Debts}
\begin{enumerate}[A]
\item If Debts are not payable\\
\textbf{Majority} Debts take priority over funeral expenses.\\
\textbf{Ḥanbalīs} Funeral expenses always take priority.
\item Debts owed to Allah ({\arabicsans{سُبْحَانَهُ وَتَعَالَىٰ}})\\
Unpaid \textit{Zakāt}, Fasting, \textit{Ḥajj}, Atonement (\textit{kafārah}) for a broken oath or \textit{naẓr}.\\
\textbf{Ḥanafīs} Abolished, because they are not people’s rights, besides heirs are not obliged to settle these  debts unless it is stated in the will. Even if stated it is only paid out of \sfrac{1}{3} of the estate.\\
\textbf{Majority} Must be paid from the estate even if it is not stated in the will.
\end{enumerate}
\end{frame}

\begin{frame}{Distribution of the Estate -- Debts}
\textbf{Shāfiʿīs} Prioritize Allah's debts over the peoples' based on the following narration.
\begin{block}{Ḥadīth \hfill al-Bukhārī and Muslim}
<INSERT HADITH HERE>
\end{block}
\end{frame}

\begin{frame}{Distribution of the Estate -- Debts}
\begin{enumerate}[A]
\setcounter{enumi}{2}
\item Debts of the deceased\\
Loan, dower or any other debts provided there is evidence.
\item Debts acknowledged during terminal illness\\
These are considered the weakest of claims to the estate. They may be gifts but they are subject to the rulings of \textit{Wasiyyah}.
\end{enumerate}
\end{frame}

\begin{frame}{Payment of Debts}
\begin{enumerate}
\item From claims owed to the deceased
\item From available cash
\item By selling the movables
\item By selling the immovables\\
\begin{itemize}
\item The estate should be distributed among the creditors in proportion to their claims.
\item Though the heirs are not personally liable to pay off the debts they still may settle by themselves and the creditors have no right to object it. 
\item However, the creditors have right to nullify any transaction of the heirs which would jeopardize the payment of their debts.
\end{itemize}
\end{enumerate}
\end{frame}

\begin{frame}{Payment of Debts}
\begin{itemize}
\item After the payment of debts the heirs have right to nullify any unfair transaction or legacy which is more than \sfrac{1}{3} of the estate made by their father or mother during his or her death illness (\textit{waqf}, gifts, sale, purchase etc).
\item Finally, only the residue is distributed among the heirs.
\item If no estate is left the unpaid debts will remain unredeemed. In this case, the creditor will either forgo or claim his debts in the Hereafter.
\end{itemize}
\end{frame}

\begin{frame}{}
\section{Inheritance \hfill \arabicsans{الميراث}}
\end{frame}

\begin{frame}{Inheritance \hfill \arabicsans{الميراث}}
\textbf{In pre-Islamic period}\\
\begin{itemize}
\item No inheritance for maternal and uterine relatives only for male agnates capable to protect the honor of the family. So, all females, minors, infirm and old were excluded.
\item A person could bequeath all his wealth to one of his heirs or to any stranger.
\end{itemize}
\textbf{In English Law}\\
Children and descendants excluded all ascendants and collaterals. Males were preferred to females and the rule of primogeniture was recognized.

\textbf{In Jewish Law}\\
Preference was given to son while parents and spouse were excluded. Besides, wives were considered as property, therefore, inherited by the heirs.
\end{frame}

\begin{frame}{\textit{Irth} or \textit{Farāʿiḍ} \hfill \arabicsans{الإرث أو علم الفرائض}}
\begin{itemize}
\item Literally: \textit{Irth} means remainder.
\item Technically: It refers to legal (\textit{fiqhī}) and mathematical rules through which the shares of each heir in the estate of the deceased is determined.
\item It is an involuntary devolution.
\item The Quran states,\\
<INSERT AYAH HERE>\\
<INSERT TRANSLATION HERE>
\item A ḥadīth narration states,\\
\hfill {\arabicfont{يأ أبا هريرة تعلموا الفرائض وعلموها فإنه نصف العلم}}\\
“O Abu Hurayrah! Learn \textit{farāʿiḍ} and teach it (to others), it is half of knowledge.”\hfill--Narrated by Ibn Mājah
\end{itemize}
\end{frame}

\begin{frame}{Pillars of Inheritance}
\begin{enumerate}
\item \arabicsans{المُورِّث} The Deceased
\item \arabicsans{الوارث} The Heirs
\item \arabicsans{الموروث أو التركة} The estate of the deceased
\end{enumerate}
\end{frame}

\begin{frame}{Pillars of Inheritance -- The Deceased \hfill \arabicsans{المُورِّث}}
\begin{itemize}
\item Actual death (\arabicsans{حقيقي}) or (\arabicsans{حكمي}) based on decree of the court in case of a missing person. Consequently, the right to inherit him will take place only on the day of the decree but not from the day of his disappearance.\\
\textbf{Shāfiʿīs} Missing person is considered alive and can inherit.\\
\textbf{Ḥanafīs} He is alive but neither he inherits nor is inherited.\\
\item The apostate is also considered dead by decree but neither he inherits nor he is inherited.
\end{itemize}
\end{frame}

\begin{frame}{Pillars of Inheritance -- The Heirs \hfill \arabicsans{الورثة}}
\begin{itemize}
\item Must survive the deceased or is deemed to have survived him as a child born alive later.
\item A child in the womb is treated as an existing child, so the other heirs may distribute the estate by reserving the share of male child provided he is born alive.
\item If more than one die in common calamity (earthquake...etc) and no possibility to prove who died first then the mutual inheritance is ignored and the estate will be distributed among the heirs of each respectively.
\end{itemize}
\end{frame}

\begin{frame}{Pillars of Inheritance -- The Estate of the Deceased \hfill\arabicsans{الموروث أو التركة}}
\begin{itemize}
\item Property, debts and any pecuniary right (preemption, retaliation, mortgaged property, usufruct).
\item Some other properties which come after his death such as dividends from his shares in a company are also included in the estate.
\item Must be valuable property in the eyes of Sharīʿah (not valuable if stolen, usurped or by any dishonest means).
\item Mortgaged property must be first released and distributed among the heirs but if the estate is not enough then it should be sold to satisfy the mortgagee’s debt.
\item Pension, gratuity, EPF fund and policy monies from Insurance are also included in the estate.
\item Compensation from employer, negligent driver, airlines should be distributed among the heirs based on law of \textit{Diyah} (blood money) in Islam.
\end{itemize}
\end{frame}

\begin{frame}{Deprivation from Inheritance}
\begin{enumerate}
\item Difference in Religion
\item Murder
\item Slavery
\item \textit{Liʿān} (Oath of Condemnation)
\item \textit{Zinā}, Adultery
\item \textit{Riddah} (Apostasy)
\item \textit{Dār al-Ḥarb} (State that is at war with the Muslims)
\end{enumerate}
\end{frame}

\begin{frame}{Principal Classes of Heirs}
\begin{enumerate}
\item Quranic Heirs or Sharers (\arabicsans{})
\item Agnatic Heirs or Residuaries (\arabicsans{})
\item \textbf{Ḥanafīs} Distant kindred (\arabicsans{})
\end{enumerate}
\end{frame}

\begin{frame}{Principal Classes of Heirs -- Quranic Heirs}
There are twelve (12) groups:\\
\textbf{Spouses}: Husband and Wife\\
\textbf{Ascendants}: Father, Mother, True Grandfather, True Grandmother\\
\textbf{Descendants}: Daughter and Sons\\
\textbf{Collaterals}: Full sister, consanguine sister, uterine sister and uterine brother.
\end{frame}

\begin{frame}{Principal Classes of Heirs -- Quranic Heirs \hfill Spouses}
<INSERT AYAH>\\
<INSERT TRANSLATION>

\begin{itemize}
\item \textbf{Husband} \sfrac{1}{2} or \sfrac{1}{4} (<CLARIFY THIS>)
\item \textbf{Wife} \sfrac{1}{4} or \sfrac{1}{8} (<CLARIFY THIS>)
\item \textbf{Wives} share among themselves either \sfrac{1}{4} or \sfrac{1}{8}
\end{itemize}
\end{frame}

\begin{frame}{Principal Classes of Heirs -- Quranic Heirs \hfill Ascendants}
True grandfather and True grandmother are substitutes if parents are unavailable.\\
<INSERT AYAH>\\
<INSERT TRANSLATION>\\
\begin{itemize}
\item \textbf{Father} \sfrac{1}{6} (if the deceased left behind a son), however, if the deceased left a female (daughter or son's daughter), he may inherit as a residuary as well.
\item \textbf{Mother} \sfrac{1}{6} if the deceased left behind a child, however, she will inherit \sfrac{1}{3} if none of the following is left behind: child, descendant, more than one brother or sister (full, consanguine, uterine)
\end{itemize}
\end{frame}